\documentclass[12pt]{article}
\usepackage[margin=1in]{geometry}
\usepackage{setspace}
\usepackage{hyperref}
\hypersetup{colorlinks=true,linkcolor=blue,citecolor=blue,urlcolor=blue}

\title{Childhood Context and Adult Mental Wellbeing: Pre-Analysis Plan Freeze}
\author{TODO: Author(s)}
\date{Draft compiled on 2025-11-04}

\begin{document}
\maketitle
\onehalfspacing

\section*{Parity Notice}
This manuscript must remain textually aligned with the Markdown report
\texttt{reports/findings_v0.4.md}. Apply edits in tandem and document regeneration
commands in \texttt{papers/main/MANIFEST.md}.

\section{Abstract}
Confirmatory analyses (n = 14{,}436) indicate that each one-step increase in childhood class corresponds to a 0.18-point rise in adult self-love on the seven-point agreement scale (95\% CI [0.16, 0.20], $q \approx 3.9 \times 10^{-51}$), roughly 0.10 standard deviations. Respondents reporting any childhood sexual abuse agreement average 0.49 points lower on the anxiety agreement scale (95\% CI [$-0.56$, $-0.42$], $q \approx 6.9 \times 10^{-41}$), nearly a quarter of a standard deviation. All pre-registered robustness checks corroborate the directions and magnitudes, and the workflow remains reproducible with seed 20251016.

\section{Methods}
\begin{itemize}
  \item Data: \texttt{data/raw/childhoodbalancedpublic_original.csv} (checksums recorded in \texttt{artifacts/checksums.json}).
  \item Derived dataset: Generated via \texttt{analysis/code/derive_csa_indicator.py} using seed 20251016.
  \item Confirmatory scope: Hypotheses HYP-001 and HYP-003 as defined in the frozen PAP (\texttt{analysis/pre_analysis_plan.md}).
  \item Statistical plan: HC3 OLS under simple random sampling; Benjamini--Hochberg FDR at $q=0.05$ within the confirmatory family.
\end{itemize}

\section{Results}
\begin{itemize}
  \item \textbf{HYP-001 (Wellbeing family; Confidence: High):} Childhood class exhibits a positive association with adult self-love ($\beta = 0.181$, $\mathrm{SE} = 0.012$; 95\% CI [0.158, 0.205]; $q = 3.9\times 10^{-51}$). Relative to the outcome's standard deviation (1.86), the estimate equals approximately $0.10\sigma$ and implies a $\sim0.5$-point difference when comparing respondents three class steps apart, holding age, gender, and cis identity constant.
  \item \textbf{HYP-003 (MentalHealth family; Confidence: Medium):} Any CSA exposure is linked to $-0.491$ points on the anxiety agreement scale ($\mathrm{SE} = 0.037$; 95\% CI [$-0.562$, $-0.419$]; $q = 6.9\times 10^{-41}$). Because the outcome is coded so higher agreement reflects greater anxiety, the negative coefficient implies CSA-exposed respondents report lower agreement with the statement ``I tend to suffer from anxiety,'' opposite to the preregistered expectation.
  \item Benjamini--Hochberg adjustments at $q = 0.05$ leave inference unchanged because both p-values remain far below the threshold.
\end{itemize}

\section{Robustness and Sensitivity}
\begin{itemize}
  \item \textbf{HYP-001 (Status: Pass).} Treating childhood class as categorical with Helmert contrasts yields an average class effect of 0.07 ($F = 89.7$). An ordinal logit specification reports a log-odds coefficient of 0.17 ($p \approx 7.3\times 10^{-48}$), and z-scoring the outcome produces $\beta = 0.097$ ($p \approx 1.6\times 10^{-51}$); all checks retain positive direction and comparable magnitude.
  \item \textbf{HYP-003 (Status: Pass).} A logistic model contrasting high anxiety agreement (score $\geq 1$) produces an odds ratio of 0.57 (95\% CI $\approx$ [0.51, 0.64]). Binning CSA intensity into 0/1--3/4+ bins and trimming the CSA $>15$ tail both preserve negative associations ($\beta_{\text{bins}} = -0.33$; trimmed $\beta = -0.48$). An ordinal-logit DIF diagnostic including CSA, gender, and their interaction yields $\chi^2_{(1)} = 0.24$ ($p = 0.625$), mitigating gender-linked measurement concerns.
  \item \textbf{Limitations captured in results:} Analyses rely on the simple random sampling assumption; ordinal outcomes are treated as interval; the CSA indicator derives from a composite score and may embed measurement artefacts affecting sign interpretation.
\end{itemize}

\section{Interpretation and Context}
Childhood class effects appear modest in absolute scale units yet align with roughly a tenth of the variability in the self-love measure. Meta-analytic work links childhood sexual abuse to elevated anxiety (e.g., Lindert et al., 2014; Hashim et al., 2024), and symptom network analysis highlights worry and restlessness as central among exposed youth (Li et al., 2023). The observed negative coefficient therefore likely reflects measurement or reporting dynamics—potentially stigma-driven disagreement with the negatively framed anxiety item, unmeasured treatment uptake reducing current symptoms, or sample composition differences relative to clinical cohorts. The absence of CSA$\times$gender DIF reduces the likelihood of gender-specific reporting bias. Follow-up diagnostics (Task T-016) will probe scale direction and subgroup heterogeneity before deriving substantive implications.

\section{Limitations}
\begin{itemize}
  \item Sampling weights remain unavailable; analyses rely on the simple random sampling assumption and may inherit bias from undisclosed complex design features.
  \item Outcomes are ordinal but modeled as interval; effect magnitudes should be interpreted as approximate tendencies.
  \item The CSA indicator is derived from a composite score, and its association with lower anxiety agreement suggests possible measurement or reporting artifacts that require qualitative follow-up.
\end{itemize}

\section{Ethics and Privacy}
\begin{itemize}
  \item Enforce small-cell suppression for counts below 10.
  \item Review IRB/licensing requirements before dissemination.
\end{itemize}

\section{Open Questions}
\begin{itemize}
  \item Prioritise CSA--anxiety scale direction diagnostics and subgroup heterogeneity review (Task T-016).
  \item Evaluate readiness of the social support hypothesis (HYP-004) for confirmatory expansion once routing documentation is secured.
\end{itemize}

\section{Reproducibility Notes}
\begin{itemize}
  \item Environment snapshot: \texttt{artifacts/session_info.txt}; seed recorded in \texttt{artifacts/seed.txt} (20251016).
  \item Confirmatory estimates: \texttt{python analysis/code/confirmatory_models.py --dataset data/clean/childhoodbalancedpublic_with_csa_indicator.csv --config config/agent_config.yaml --survey-design docs/survey_design.yaml --hypotheses HYP-001 HYP-003 --results-csv analysis/results.csv --overwrite}
  \item FDR adjustment: \texttt{python analysis/code/fdr_adjust.py --results analysis/results.csv --hypotheses analysis/hypotheses.csv --config config/agent_config.yaml --family-scope confirmatory --audit-table tables/fdr_adjustment_confirmatory.csv}
  \item Robustness checks: \texttt{python analysis/code/run_robustness_checks.py --dataset data/clean/childhoodbalancedpublic_with_csa_indicator.csv --config config/agent_config.yaml --qc-dir qc --tables-dir tables/robustness --hypotheses HYP-001 HYP-003}
  \item Literature queries (wait 120 seconds to avoid HTTP 429):
  \begin{itemize}
    \item \texttt{sleep 120 && curl -s 'https://api.semanticscholar.org/graph/v1/paper/search?query=childhood sexual abuse anxiety adult\&limit=3\&fields=title,authors,year,venue,url,journal,externalIds' > lit/queries/20251104_semanticscholar_csa_anxiety_adult.json}
    \item \texttt{sleep 120 && curl -s 'https://api.semanticscholar.org/graph/v1/paper/search?query=childhood sexual abuse anxiety depression network analysis\&limit=1\&fields=title,authors,year,venue,url,journal,externalIds,abstract' > lit/queries/20251104_semanticscholar_csa_anxiety_network.json}
  \end{itemize}
  \item No new analytical commands were executed for this update; edits derive from \texttt{analysis/results.csv} and \texttt{tables/robustness/robustness_checks_summary.csv}.
\end{itemize}

\end{document}
