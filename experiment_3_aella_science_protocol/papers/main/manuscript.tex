\documentclass[12pt]{article}
\usepackage[margin=1in]{geometry}
\usepackage{setspace}
\usepackage{hyperref}
\hypersetup{colorlinks=true,linkcolor=blue,citecolor=blue,urlcolor=blue}

\title{Childhood Context and Adult Mental Wellbeing: Pre-Analysis Plan Freeze}
\author{TODO: Author(s)}
\date{Draft compiled on 2025-11-04}

\begin{document}
\maketitle
\onehalfspacing

\section*{Parity Notice}
This manuscript must remain textually aligned with the Markdown report
\texttt{reports/findings_v0.1.md}. Apply edits in tandem and document regeneration
commands in \texttt{papers/main/MANIFEST.md}.

\section{Abstract}
Confirmatory analyses (n = 14{,}436) indicate that each one-step increase in childhood class corresponds to a 0.18-point rise in adult self-love on the seven-point scale (95\% CI [0.16, 0.20], $q \approx 3.9\times 10^{-51}$). Contrary to the preregistered direction, respondents reporting any childhood sexual abuse agreement average 0.49 points lower on the anxiety agreement scale (95\% CI [$-0.56$, $-0.42$], $q \approx 6.9\times 10^{-41}$). All estimates rely on HC3 OLS under a simple random sampling assumption and are reproducible with seed 20251016.

\section{Methods}
\begin{itemize}
  \item Data: \texttt{data/raw/childhoodbalancedpublic_original.csv} (checksums recorded in \texttt{artifacts/checksums.json}).
  \item Derived dataset: Generated via \texttt{analysis/code/derive_csa_indicator.py} using seed 20251016.
  \item Confirmatory scope: Hypotheses HYP-001 and HYP-003 as defined in the frozen PAP (\texttt{analysis/pre_analysis_plan.md}).
  \item Statistical plan: HC3 OLS under simple random sampling; Benjamini--Hochberg FDR at $q=0.05$ within the confirmatory family.
\end{itemize}

\section{Results}
\begin{itemize}
  \item \textbf{HYP-001 (Wellbeing family):} Childhood class exhibits a positive association with adult self-love ($\beta = 0.181$, $\mathrm{SE} = 0.012$; 95\% CI [0.158, 0.205]; $q = 3.9\times 10^{-51}$), implying roughly a fifth of a scale-point increase per class step when covarying age, gender, and cis identity.
  \item \textbf{HYP-003 (MentalHealth family):} Any CSA exposure is linked to $-0.491$ points on the anxiety agreement scale ($\mathrm{SE} = 0.037$; 95\% CI [$-0.562$, $-0.419$]; $q = 6.9\times 10^{-41}$). The negative sign reflects lower agreement with the statement ``I tend to suffer from anxiety,'' running counter to the preregistered positive direction.
\end{itemize}
Benjamini--Hochberg adjustments at $q = 0.05$ do not alter inference because the extremely small p-values remain far below the threshold.

\section{Robustness and Sensitivity}
\begin{itemize}
  \item \textbf{HYP-001:} Treating childhood class as categorical with Helmert contrasts yields an average class effect of 0.07 ($F = 89.7$). An ordinal logit specification reports a log-odds coefficient of 0.17 ($p \approx 7.3\times 10^{-48}$), and z-scoring the outcome produces $\beta = 0.097$ ($p \approx 1.6\times 10^{-51}$). All checks retain the positive direction and comparable magnitudes.
  \item \textbf{HYP-003:} A logistic model contrasting high anxiety agreement (score $\geq 1$) produces an odds ratio of 0.57 (95\% CI $\approx$ [0.51, 0.64]). Binning CSA intensity into 0/1--3/4+ bins and trimming the CSA $>15$ tail both keep negative associations ($\beta_{\text{bins}} = -0.33$; trimmed $\beta = -0.48$), corroborating the observed direction despite diverging from the preregistered expectation.
\end{itemize}

\section{Limitations}
\begin{itemize}
  \item Sampling weights remain unavailable; analyses rely on the simple random sampling assumption and may inherit bias from undisclosed complex design features.
  \item Outcomes are ordinal but modeled as interval; effect magnitudes should be interpreted as approximate tendencies.
  \item The CSA indicator is derived from a composite score, and its association with lower anxiety agreement suggests possible measurement or reporting artifacts that require qualitative follow-up.
\end{itemize}

\section{Ethics and Privacy}
Enforce small-cell suppression for counts below 10 and review IRB/licensing requirements before dissemination.

\section{Open Questions}
\begin{itemize}
  \item Evaluate readiness of the social support hypothesis (HYP-004) for future PAP inclusion.
  \item Determine whether the negative CSA--anxiety association stems from coding (negative vs.\ positive framing) or differential nonresponse requiring weighting or qualitative context.
\end{itemize}

\end{document}
